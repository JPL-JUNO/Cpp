\chapter{程序的组成部分}
\section{程序的组成部分}
\subsection*{预处理器编译指令}
C++ 编译器执行的第一项操作是,调用另一个被称为预处理器的工具对源代码进行检查,这是在编译器每次运行时自动进行的。

如果第一个字符是符号 \#,它指出这行是一个将由预处理器处理的命令。这些命令称为预处理器编译指令。预处理器的职责是,阅读代码,查找编译指令并根据编译指令相应地修改代码。修改后的代码将提供给编译器。

预处理器相当于编译前的代码编辑,每条编译指令都是一个命令,告诉这位编辑如何做。编译指令 \#include 告诉预处理器,将指定文件的全部内容加入到程序的指定位置。C++ 提供了一个标准源代码库,你可在程序中使用它
们来执行有用的功能。文件 iostream 中的代码支持输入输出任务,如
在屏幕上显示信息以及从用户那里接受输入。

文件名 iostream 前后的 $<>$ 告诉预处理器,前往一组标准位置寻找该文件。由于这些尖括号,预处理器将前往为编译器存储头文件的目录中查找文件 iostream。这些文件也被称为包含文件,因为它们被包含在源代码中。
\subsection*{源代码行}
函数是执行一个或多个相关操作的代码块,它执行某些操作后返回到调用它的位置。

每个 C++ 程序都包含一个 main() 函数,程序运行时将自动调用 main()。在 C++ 中,所有函数都必须在完成任务后返回一个值。函数 main() 总是返回一个整数,这是使用关键字 int 指定的。

与 C++ 程序中的其他代码块一样,函数也包含在 \{ 和 \} 内。所有函数都以左大括号 \{ 开头,并以右大括号 \} 结尾。

std::cout 后面是 $<<$,它被称为输出重定向运算符。运算符是代码行中根据某种信息执行操作的字符。运算符 $<<$ 显示它后面的信息。

通常,程序返回 0 表示它运行成功,而其他数字表示出现了某种故障。
\section{注释}
在 C++ 中,有两种类型的注释。单行注释以两个斜杆(//)打头,导致编译器忽略从这里开始到行尾的全部内容。

多行注释以斜杠和星号(/*)打头,并以星号和斜杆(*/)结尾。/* 和 */ 之间的所有内容都是注释,哪怕它们占据多行。如果程序中不存在与 */ 配对的 /*,编译器将视之为错误。
\begin{tcolorbox}[title=警告]
    关于多行注释,需要牢记的一个重点是,不能将其嵌套。如果你使用 /* 开始注释,并在几行后又使用了一个 /*,则编译器见到第一个 */ 后,将认为多行注释到此结束,这样第二个 */ 将导致编译器错误。
\end{tcolorbox}
\section{函数}
main() 是独特的 C++ 函数,因为程序启动时将自动调用它。

程序从函数 main() 开头开始,逐行执行源代码。调用函数时,程序将转而执行该函数,函数执行完毕后,将返回到调用函数的代码行。函数可能返回值,也可能不返回,但函数 main() 是个例外,它总是返回一个整数。

函数由函数头和函数体组成,其中函数头包含以下三项内容。
\begin{itemize}
    \item 函数的返回类型。
    \item 函数名。
    \item 函数接受的参数。
\end{itemize}

函数名是一个简短的标识符,描述了函数的功能。

函数不返回值时,使用返回类型void,这表示空。

参数是传递给函数的数据,控制函数做什么,函数收到的参数称为\textbf{实参}。函数可接受零个、一个或多个参数。参数放在括号内,用逗号分隔,构成参数列表。没有参数的函数包含一组空括号。

函数的名称、参数及其排列顺序被称为签名,函数的签名也唯一地标识了它。

函数名不能包含空格,一般采用驼峰命名法,除第一个单词外,其他每个单词的首字母都大写。

函数体由左大括号、零或多条语句以及右大括号组成。返回值的函数使用 return 语句。return 语句导致函数结束。如果函数不包含 return 语句,将自动在函数体末尾返回 void。在这种情况下,必须将函数的返回类型指定为 void。