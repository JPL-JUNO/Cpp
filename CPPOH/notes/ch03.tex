\chapter{创建变量和常量}
\section{变量是什么}
变量是计算机内存中的一个位置,您可在这里存储和检索值。可将计算机内存视为一系列排成长队的文件架,并按顺序都为每个文件架进行了编号,而文件架的编号就相当于内存地址。

变量有地址,并赋予了描述其用途的名称。变量名相当于文件架上的标签,这样无需知道变量的实际内存地址就能访问它。
\figures{fig3-1}{图显示了 7 个文件架,它们的地址为 101-107。在文件架 104 中,变量 zombies 的值为 17,而其他文件架是空的。}

\subsection*{在内存中存储变量}
在 C++ 中,当您创建变量时,必须将变量的名称和存储的信息类型(如整数、字符或浮点数)告诉编译器,这就是变量的类型,有时也称为\textbf{数据类型}。通过变量的类型,编译器将知道需要预留多少内存空间,以存储变量的值。

内存中的每个文件架为 1 字节,如果变量长 2 字节,将需要 2 字节的内存。

短整型(在 C++ 中用 short 表示)通常占用 2 字节,长整型(long)占用 4 字节,整型(int)可以为 2 或 4 字节,长长整型(long long)为 8 字节。

字符类型(在 C++ 中用 char 表示)通常为 1 字节。在 \autoref{fig3-1} 中,每个文件架表示 1 字节,因此一个短整型变量可能占据文件架 106 和 107。

布尔值用 bool 类型变量存储,这种变量只能存储值 true 或 false。

short 的长度总是不超过 int,而 int 的长度总是不超过 long。浮点数类型与此不同。

前面提及的常见变量类型的长度并不适用于所有系统,要获悉变量类型的长度,可使用函数 sizeof(),并在括号内指定类型名。函数 sizeof() 由编译器提供,不需要使用编译指令 include。
\subsection*{无符号变量和带符号变量}
所有整型变量都又分两种类型,这是使用一个关键字指定的。当这些变量只存储正值时,可使用关键字 unsigned 进行声明,而当它们可存储正值或负值时,使用关键字 signed 进行声明。无符号整型变量和带符号整型变量都可存储 0。如果声明时没有指定,则默认为带符号的。带符号和无符号整型变量占据的字节数相同,因此,无符号整型变量可存储的最大数是带符号整型变量可存储的最大正数的两倍。
\subsection*{变量类型}
除整型变量外,C++ 还支持浮点数类型和字符。

浮点变量可存储包含小数的值,而字符变量占用 1 字节,可存储 ASCII 字符集中的 256 个字符和符号之一。

在 C++ 中,short 和 long 变量也称为 short int 和 long int,在程序中这两种名称都可以使用。

可将 char 变量用于存储很小的整数,但这是一种糟糕的编程习惯。每个字符都有相应的数值,即字符集中的 ASCII 码。
\section{定义变量}
在 C++ 中,变量是通过声明其类型和名称定义的,并以分号结束语句。在一条语句中可定义多个变量,只要它们的类型相同。在这种情况下,应用逗号将变量名分隔。

C++ 保留了一些单词,不能将它们用作变量名,因为它们是 C++ 使用的关键字。保留的关键字包括 if、while、for 和 main。通常,任何合理的变量名都几乎不是关键字。
\section{给变量赋值}
要给变量赋值,可使用运算符 =,它被称为赋值运算符。可在声明变量的同时给它赋初值。这称为初始化变量。初始化看起来像赋值,但后面使用常量时,将看到有些变量必须初始化,因为不能给它们赋值。

\section{使用类型定义}
当 C++ 程序包含大量变量时,不断输入原本的变量类型,比如 unsigned short 既繁琐又容易出错。要创建现有类型的简捷表示,可使用关键字 typedef,它表示类型定义(type definition)。

类型定义要求使用关键字 typedef,后面跟现有类型及其新名称。
\section{常量}
与变量一样,常量也是一个内存位置,可在其中存储值;不同的是,常量的值不会改变,您必须在创建常量时对其进行初始化。C++ 支持两种类型的常量:字面常量和符号常量。

字面常量是直接在需要的地方输入的值。符号常量是用名称表示的常量,与变量类型相似。声明符号常量时,需要使用关键字 const,并在后面跟类型、名称和初值。通常将常量名全部大写,以区分于变量。
\subsection*{定义常量}
还有另一种定义常量的方法,这起源于早期的 C 语言(C++ 的前身)版本。可使用编译指令 \#define 来创建常量,方法是在它后面指定常量的名称和值,并用空格将它们分开。

这种常量不需要指定类型,如 int 或 char。编译指令 \#define 执行简单的文本替换,比如将代码中所有的 KILLBONUS 都替换为 5 000,编译器只能看到替换后的结果。

由于这种常量没有指定类型,因此编译器无法确保它们的值是合适的。
\subsection*{枚举常量}
枚举常量在一条语句中创建一组常量,它们是使用关键字 enum 定义的,后面跟一组用逗号分隔的名称,这些名称放在大括号内。

枚举常量值以 0(对应于第一个常量)打头,其他常量的值依次加 1。

也可以使用赋值运算符指定枚举常量的值。这种方法的优点是,可使用符号名称,如 BLACK 和 WHITE,而不是无意义的数字,如 1 或 700。
\section{自动变量}
C++ 还有一个关键字 auto,可用于根据赋给变量的初值推断出变量的类型,这种工作使用编译器完成的。使用 auto 来声明变量时,必须同时对变量进行初始化。

可使用一个 auto 关键字来声明多个变量,条件是这些变量的数据类型相同。
\begin{tcolorbox}[title=警告]
    在较久的 C++ 版本中,关键字 auto 用于指出变量为程序中的局部变量--这种概念被称为作用域。C++ 标准制定者研究了数百万行代码,发现 auto 的这种用法很少见,它主要用于测试包中。因此,他们认为 auto 的这种含义是多余的,进而赋予他用于自动确定变量类型的新功能。如果代码以以前的方式使用了 auto,则在 C++14 中将行不通。
\end{tcolorbox}
\section{总结}
变量用于存储在程序运行过程中可改变的值,而常量用于存储不变的值,换句话说,它们是不可变的。

使用变量时,最大的挑战在于选择合适的类型。如果处理的无符号整数可能大于 65000,就应将其存储在 long 变量而不是 short 变量中;如果它们可能大于 21 亿,对 long 类型来说,就太大了。如果数字值包含小数部分,就必须使用 float 或 double 变量来存储,这是 C++ 支持的两种浮点类型。

使用变量时,要牢记的另一点是它们占用的字节数,这因系统而异。函数 sizeof() 提供了编译器返回的变量类型占用的字节数。