\chapter{使用表达式、语句和运算符}
\section{语句}
所有 C++ 都由语句组成,语句是以分号结尾的命令。根据约定,每条语句占一行,但并非必须这样:可将多条语句放在一行,只要每条语句都以分号结尾即可。语句控制程序的执行流程、评估表达式甚至可以什么也不做(空语句)。
\subsection*{空白}
在 C++ 程序的源代码中,空格、制表符和换行符统称为空白。空白旨在让程序员方便阅读代码,编译器通常忽略它们。

编译器忽略空白,变量名不能包含空白。
\subsection*{复合语句}
可将多条语句编组,构成一条复合语句,这种语句以左大括号 \{ 开头,以右大括号 \} 结束。可将复合语句放在任何可使用单条语句的地方。

复合语句中的每条语句都必须以分号结尾,但复合语句本身不能以分号结尾,
\section{表达式}
表达式是语句中任何返回一个值的部分。赋值运算符 = 导致左操作数的值变为右操作数的值。操作数是一个数学术语,指的是被运算符操作的表达式。
\section{运算符}
运算符是导致编译器执行操作的符号,如赋值、执行乘法运算、除法运算或其他数学运算。
\subsection*{赋值运算符}
赋值表达式由赋值运算符、左操作数(也叫左值)和右操作数(也叫右值)组成。

常量可作为右值,但不能作为左值。左值和右值可能出现在编译器错误消息中。

\subsection*{数学运算符}
数学运算符有五个:加法运算符(+)、减法运算符(-)、乘法运算符(*)、除法运算符(/)和求模运算符(\%)。与 C 语言一样,C++ 也没有乘方运算符(将一个值相乘指定次数),这种任务由函数完成。

加法、减法和乘法运算符与您预期的一致,但除法运算符更复杂。

\textbf{整数除法与普通除法不同}。将 21 除以 4 时,结果是一个带小数的实数。但整数除法的结果为整数,余数被丢弃,因此 21 / 4 返回5。

计算模数在编程中很有用。如果要在任务每执行 10 次就显示一条声明,就可以使用表达式 Count \% 10。在这里,模数的范围为 0-9,每当模数为 0,就说明任务执行的次数为 10 的整数倍。

浮点数除法与常规除法相同,表达式 21 / 4.0 的结果为 5.25。

C++ 根据操作数的类型决定执行哪种除法。\textbf{只要有一个运算符为浮点数变量或浮点数字面量,就将执行浮点数除法;否则,执行整数除法。}
\subsection*{组合运算符}
经常需要将一个变量与一个值相加,并将结果赋给这个变量。自赋值加法运算符+=将右值与左值相加,然后将结果赋给左值。还有自赋值减法运算符(−=)、自赋值除法运算符(/=)、自赋值乘法运算符(*=)和自赋值求模运算符(\%=)。

\subsection*{递增和递减运算符}
将变量加 1 或减 1 很常见。将变量加 1 称为递增,而将变量减 1 称为递减。C++ 提供了递增运算符 ++ 和递减运算符 - -,用于完成这些任务。
\subsection*{前缀运算符和后缀运算符}
递增运算符 ++ 和递减运算符 - - 可放在变量名前面,也可放在变量名后面,但效果不同。放在变量名前面称为前缀运算符,放在变量名后面称为后缀运算符。

将变量递增或递减,再将结果赋给另一个表达式时,前缀运算符和后缀运算符的差别将显现出来:\textbf{后缀运算符在赋值后执行。}
\subsection*{运算符优先级}
复杂表达式的结果取决于运算符优先级,即表达式的计算顺序。每个运算符都有优先级。表 \autoref{tbl4-1} 列出了运算符优先级。\textbf{突然没看懂这里面的函数,是不是有很多重复的,但是优先级不同?}
\begin{table}
    \centering
    \caption{运算符优先级}
    \label{tbl4-1}
    \begin{tabular}{lll}
        \hline
        优先级    & 运算符                                           & 运算顺序 \\
        \hline
        1(最高)  & () . [] $\rightarrow$ ::                      & 从左到右 \\
        2      & * \& ! $\sim$ ++ - - + -                      & 从右到左 \\
               & sizeof new delete                             & 从左到右 \\
        3      & .* $\rightarrow$ *                            & 从左到右 \\
        4      & * /                                           & 从左到右 \\
        5      & + -                                           & 从左到右 \\
        6      & $<<$ $>>$                                     & 从左到右 \\
        7      & $<$ $<=$ $>$ $>=$                             & 从左到右 \\
        8      & == !=                                         & 从左到右 \\
        9      & \&                                            & 从左到右 \\
        10     & $\hat{}$                                      & 从左到右 \\
        11     & $|$                                           & 从左到右 \\
        12     & \&\&                                          & 从左到右 \\
        13     & $||$                                          & 从左到右 \\
        14     & ?:                                            & 从右到左 \\
        15     & = *= /= += \%= $<<=$ $>>=$ \&= $\hat{}=$ $|=$ & 从右到左 \\
        16(最低) & .                                             & 从左到右 \\
        \hline
    \end{tabular}
\end{table}

当优先级顺序不符合要求时,可使用括号来改变顺序。括号内运算符的优先级比其他任何数学运算符都高。括号可以嵌套,在这种情况下,将首先计算最内面的括号内的表达式。

\subsection*{关系运算符}
关系运算符用于比较,以判断一个数是大于、等于还是小于另一个数。每个关系表达式都是要么返回 true,要么返回 false。\autoref{tbl4-2} 列出了关系运算符。

\begin{table}
    \centering
    \caption{关系运算符}
    \label{tbl4-2}
    \begin{tabular}{ll}
        \hline
        名称   & 运算符  \\
        \hline
        相等   & ==   \\
        不等   & !=   \\
        大于   & $>$  \\
        大于等于 & $>=$ \\
        小于   & $<$  \\
        小于等于 & $<=$ \\
        \hline
    \end{tabular}
\end{table}
\section{if-else 条件语句}
通过使用关键字 if,可在仅满足条件时执行代码。
\subsection*{else子句}
程序可在 if 条件为 true 时执行一条语句,并在 if 条件为 false 时执行另一条语句。要指定 if 条件为 false 执行的语句,可使用关键字 else。
\subsection*{复合if语句}
在可使用单条语句的任何地方,都可使用复合语句。if 条件和 if-else 条件后面通常是复合语句。可将任何语句与 if 条件结合使用,这包括另一个 if 条件子句,甚至另一条 if 或 else 语句。

\section{逻辑运算符}
通过使用逻辑运算符,可测试多个条件,这包括与运算符 \&\& 和或运算符 $||$,而逻辑运算符非 ! 检查表达式是否为 false。
\subsection*{与运算符}
逻辑运算符与连接两个表达式,如果它们都为 true,那么整个表达式的结果也为 true。
\subsection*{或运算符}
逻辑运算符或连接两个表达式,只要这两个表达式之一为 true,整个表达式就为 true。
\subsection*{非运算符}
逻辑非运算符对表达式求反,在表达式为 false 时返回 true,而在表达式为 true 时返回 false。
\subsection*{关系运算符和逻辑运算符的优先级}
逻辑运算符与和或的优先级相同,因此按从左到右的顺序计算。
\section{棘手的表达式值}
条件表达式的结果为 true 或 false。在 C++ 中,0 也被认为是 false,而其他值被认为是 true。