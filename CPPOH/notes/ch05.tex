\chapter{调用函数}
\section{函数是什么}
函数是程序的一部分,可对数据执行操作并返回一个值。每个 C++ 程序都至少有一个函数:程序运行时自动调用的函数 main()。这个函数可包含调用其他函数的语句,而这些函数中有些可能又调用其他函数,以此类推。

每个函数都有名称,可用于调用它。函数被调用时,将首先执行其第一条语句,再不断执行,直到到达最后一条语句,然后返回到调用该函数的地方执行。

设计良好的函数执行特定任务。对于复杂的任务,应将其划分成多个函数,然后依次调用它们,这让代码更容易理解和维护。
\section{声明和定义函数}
编写函数的代码前,必须声明它。函数声明将函数的名称、函数返回的数据类型以及函数参数的类型告诉编译器。函数声明也叫原型,不包含任何代码。

声明将函数的工作方式告诉编译器。函数原型是单条语句,以分号结尾。

参数列表列出了所有参数及其类型,并用逗号分隔它们。

函数原型必须与函数的三个元素匹配,否则无法通过编译;唯一无需匹配的是形参名。在函数声明中,可根本不指定形参名。

函数可返回任何 C++ 数据类型。如果函数不返回值,就应将返回类型声明为 void。返回类型为 void 的函数不需要包含 return 语句。

不同于函数声明,在函数定义中,指定函数名的语句不以分号结尾。

\begin{tcolorbox}
    如果将函数定义移到调用它的代码前面,则不需要原型。在小型程序(如本书创建的程序)中,这样可行,但在大型编程项目中,确保所有函数在使用前都进行了定义很麻烦。通过使用原型声明所有函数,就不用考虑这个问题了。
\end{tcolorbox}
\section{在函数中使用变量}
函数以多种方式使用变量:调用函数时可将变量指定为参数;在函数内部可声明变量,这些变量在函数执行完毕后将消失;还可在函数和程序的其他部分之间共享变量。
\subsection*{局部变量}
在函数内创建的变量为局部变量,因为它只存在于函数内,函数返回后,其所有局部变量都不能供程序使用。

局部变量的创建方式与其他变量相同,函数的形参也被视为局部变量。

在块中声明的变量的作用域为当前块,到达该代码块末尾的右大括号后,这些变量便不可用。可在任何代码块中声明变量,如 if 条件语句和函数内。
\subsection*{全局变量}
在 C++ 程序中,也可在函数(包括函数 main())外面定义 C++ 变量,这样的变量称为全局变量,因为它们在程序的任何地方都可用。

在函数外面定义的变量的作用域为全局,因此可在程序的任何函数(包括 main())内使用。
