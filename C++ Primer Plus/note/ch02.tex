\chapter{开始学习C++\label{ch01}}
\section{进入 C++}
\subsection{main() 函数}
C++ 和 C 一样,也是用终止符(terminator),而不是分隔符。终止符是一个分号,他是语句的计数标记,是语句的组成部分,而不是语句之间的标记。结论是:在 C++ 中,不能省略分号。
\subsubsection{作为接口的函数头}
通常,C++ 函数可被其他函数激活或调用,函数头描述了函数与调用它的函数之间的接口。位于函数名前面的部分叫做函数返回类型,它描述的是从函数返回给调用它的函数的信息。函数名后括号中的部分叫做形参列表(argument list)或参数列表(parameter list);它描述的是从调用函数传递给被调用函数的信息。

在括号中使用关键字参数 void 显示地指出,函数不接受任何参数。在 C++ 中(不是 C) 中,让括号空着与在括号中使用 void 等效(在 C 中,让括号空着意味着对是否接受参数保持沉默)。
\subsection{C++ 预处理器和 iostream 文件}
\subsection{头文件名}
\begin{table}
    \centering
    \caption{头文件命名约定}
    \label{tbl2-1}
    \begin{tabularx}{\textwidth}{lllX}
        \hline
        头文件类型    & 约定           & 示例         & 说明                                      \\
        \hline
        C++ 旧式风格 & 以 .h 结尾      & iostream.h & C++ 程序可以使用                              \\
        C 旧式风格   & 以 .h 结尾      & math.h     & C、 C++ 程序可以使用                           \\
        C++ 新式风格 & 没有拓展名        & iostream   & C++ 程序可以使用,使用 namespace std             \\
        转化后的 C   & 加上前缀 c,没有拓展名 & cmath      & C++ 程序可以使用,可以使用不是 C 的特性,如 namespace std \\
        \hline
    \end{tabularx}
\end{table}
\subsection{名称空间}
如果使用 iostream,而不是 iostream.h,则应使用下面的名称空间编译指令来使 iostream 中的定义对程序可用:
\begin{verbatim}
    using namespace std;
\end{verbatim}
\subsection{使用 cout 进行 C++ 输出}
从概念上看,输出是一个流,即从程序流出的一系列字符。cout 对象表示这种流,其属性是在iostream文件中定义的。cout 的对象属性包括一个插入运算符(\verb|<<|),它可以将其右侧的信息插入到流中。
\subsubsection{控制符 endl}
endl 是一个特殊的 C++ 符号,表示一个重要的概念:重起一行。在输出流中插入 endl 将导致屏幕光标移到下一行开头。诸如 endl 等对于 cout 来说有特殊含义的特殊符号被称为控制符(manipulator)。和 cout 一样,endl 也是在头文件 iostream 中定义的,且位于名称空间 std 中。
\section{C++ 语句}
C++ 程序是一组函数,而每个函数又是一组语句。声明语句创建变量,赋值语句给该变量提供一个值。
\subsection{声明语句和变量}
为什么变量必须声明?有些语言(最典型的是 BASIC)在使用新名称时创建新的变量,而不用显式地进行声明。这看上去对用户比较友好,事实上从短期上说确实如此。问题是,如果错误地拼写了变量名,将在不知情的情况下创建一个新的变量。

程序中的声明语句叫做\textbf{定义声明}(defining declaration)语句,简称为定义(definition)。这意味着它将导致编译器为变量分配内存空间。在较为复杂的情况下,还可能有\textbf{引用声明}(reference declaration)。这些声明命令计算机使用在其他地方定义的变量。
\subsection{cout 的新花样}
cout 的智能行为源自 C++ 的面向对象特性。实际上,C++ 插入运算符(\verb|<<|)将根据其后的数据类型相应地调整其行为,这是一个运算符重载的例子。
\section{其他 C++ 语句}
\subsection{使用 cin}
就像 C++ 将输出看作是流出程序的字符流一样,它也将输入看作是流入程序的字符流。iostream 文件将 cin 定义为一个表示这种流的对象。输出时,<<运算符将字符串插入到输出流中;输入时,cin 使用 \verb|>>| 运算符从输入流中抽取字符。通常,需要在运算符右侧提供一个变量,以接收抽取的信息(符号 \verb|<<| 和 \verb|>>| 被选择用来指示信息流的方向)。
\subsection{类简介}
类是用户定义的一种数据类型。要定义类,需要描述它能够表示什么信息和可对数据执行哪些操作。类之于对象就像类型之于变量。也就是说,类定义描述的是数据格式及其用法,而对象则是根据数据格式规范创建的实体。

\textbf{注意:类描述了一种数据类型的全部属性(包括可使用它执行的操作),对象是根据这些描述创建的实体。}

类描述指定了可对类对象执行的所有操作。要对特定对象执行这些允许的操作,需要给该对象发送一条消息。C++提供了两种发送消息的方式:一种方式是使用类方法(本质上就是函数调用);另一种方式是重新定义运算符,cin 和 cout 采用的就是这种方式。
\section{函数}
C++函数分两种:有返回值的和没有返回值的。
\subsection{使用有返回值的函数}
有返回值的函数将生成一个值,而这个值可赋给变量或在其他表达式中使用。
\figures{fig2-6}{调用函数}

表达式 \verb|sqrt(6.25)| 将调用 \verb|sqrt()|函数。表达式 \verb|sqrt(6.25)| 被称为函数调用,被调用的函数叫做被调用函数(called function),包含函数调用的函数叫做调用函数(calling function,参见\autoref{fig2-6})。
\figures{fig2-7}{函数调用的句法}

在使用函数之前,C++ 编译器必须知道函数的参数类型和返回值类型。也就是说,函数是返回整数、字符、小数、有罪裁决还是别的什么东西?如果缺少这些信息,编译器将不知道如何解释返回值。C++ 提供这种信息的方式是使用函数原型语句。

\textbf{C++ 程序应当为程序中使用的每个函数提供原型。}

函数原型之于函数就像变量声明之于变量—指出涉及的类型。sqrt()的函数原型像这样:
\begin{verbatim}
    double sqrt(double); // function prototype
\end{verbatim}
第一个 double 意味着 sqrt() 将返回一个 double 值。括号中的 double 意味着sqrt()需要一个 double 参数。因此该原型对 sqrt() 的描述和下面代码中使用的函数相同:
\begin{verbatim}
    double x; // declare x as a type double variable
    x = sqrt(6.25);
\end{verbatim}
原型结尾的分号表明它是一条语句,这使得它是一个原型,而不是函数头。如果省略分号,编译器将把这行代码解释为一个函数头,并要求接着提供定义该函数的函数体。

不要混淆函数原型和函数定义。可以看出,原型只描述函数接口。也就是说,它描述的是发送给函数的信息和返回的信息。而定义中包含了函数的代码,如计算平方根的代码。C 和 C++ 将库函数的这两项特性(原型和定义)分开了。库文件中包含了函数的编译代码,而头文件中则包含了原型。
\subsection{函数变体}
\subsection{用户定义的函数}
假设需要添加另一个用户定义的函数。和库函数一样,也可以通过函数名来调用用户定义的函数。对于库函数,在使用之前必须提供其原型,通常把原型放到 main() 定义之前。但现在您必须提供新函数的源代码。最简单的方法是,将代码放在 main() 的后面。
\subsubsection{函数格式}
注意,定义 simon() 的源代码位于main( )的后面。和 C 一样(但不同于 Pascal),C++ 不允许将函数定义嵌套在另一个函数定义中。每个函数定义都是独立的,所有函数的创建都是平等的(参见 \autoref{fig2-8})。
\figures{fig2-8}{函数定义在文件中依次出现}

main()返回一个int值,而程序员要求它返回整数0。但可能会产生疑问,将这个值返回到哪里了呢?毕竟,程序中没有哪个地方可以看出对main()的调用:
\begin{verbatim}
    squeeze = main(); // absent from our programs
\end{verbatim}
答案是,可以将计算机操作系统(如 UNIX 或 Windows)看作调用程序。因此,main() 的返回值并不是返回给程序的其他部分,而是返回给操作系统。很多操作系统都可以使用程序的返回值。例如,UNIX 外壳脚本和 Windows 命令行批处理文件都被设计成运行程序,并测试它们的返回值(通常叫做退出值)。通常的约定是,退出值为 0 则意味着程序运行成功,为非零则意味着存在问题。因此,如果 C++ 程序无法打开文件,可以将它设计为返回一个非零值。然后,便可以设计一个外壳脚本或批处理文件来运行该程序,如果该程序发出指示失败的消息,则采取其他措施。
\subsection{用户定义的有返回值的函数}
main() 函数已经揭示了有返回值的函数的格式:在函数头中指出返回类型,在函数体结尾处使用 return。

通常,在可以使用一个简单常量的地方,都可以使用一个返回值类型与该常量相同的函数。

函数原型描述了函数接口,即函数如何与程序的其他部分交互。参数列表指出了何种信息将被传递给函数,函数类型指出了返回值的类型。程序员有时将函数比作一个由出入它们的信息所指定的黑盒子(black boxes)(电工用语)。函数原型将这种观点诠释得淋漓尽致。
\subsection{在多函数程序中使用 using 编译指令}
当前通行的理念是,只让需要访问名称空间 std 的函数访问它是更好的选择。

让程序能够访问名称空间std的方法有多种,下面是其中的 4 种。
\begin{enumerate}
    \item  将 \verb|using namespace std;| 放在函数定义之前,让文件中所有的函数都能够使用名称空间 std 中所有的元素。
    \item  将 \verb|using namespace std;| 放在特定的函数定义中,让该函数能够使用名称空间 std 中的所有元素。
    \item  在特定的函数中使用类似 \verb|using std::cout;| 这样的编译指令,而不是 \verb|using namespace std;|,让该函数能够使用指定的元素,如 cout。
    \item  完全不使用编译指令 using,而在需要使用名称空间 std 中的元素时,使用前缀\verb|std::|,如下所示:
\end{enumerate}
\begin{verbatim}
    std::cout << "I'm using cout and endl from the std namespace" << std::endl;
\end{verbatim}